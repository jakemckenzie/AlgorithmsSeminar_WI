\documentclass[12pt]{article}
\usepackage{amsmath,amssymb,amsthm}
\usepackage{graphicx,mathabx}
\usepackage{xcolor}
\usepackage{tikz}
\usepackage{cases}
\usepackage{placeins}
\usepackage{lipsum}
\usepackage{multirow}
\usepackage{mathtools}
\usepackage[shortlabels]{enumitem}
\usepackage{wrapfig}
\DeclarePairedDelimiter{\floor}{\lfloor}{\rfloor}
\begin{document}
\title{TCSS 343 - Week 6 - Monday}
\author{Jake McKenzie}
\maketitle
\noindent\centerline{\textbf{Greedy Algorithsm and Dynamic Programming}}\\\\\\\\\\\\
\begin{center}
    ``A distributed system is when some computer I didn't know existed fails and causes my computer to fail." \\$\cdots$\\ Leslie Lamport
\end{center}
\begin{center}
    ``Greedy algorithms: do the thing that looks best right now, and repeat until nothing looks good anymore or you're forced to stop." \\$\cdots$\\ Jeremy Kun
\end{center}
\begin{center}
    ``We don't much care if you don't approve of the software we write." \\$\cdots$\\ Eric Hughes
\end{center}
\newpage
\includegraphics[width=\textwidth]{svenborgia.jpg}\\
    \noindent You want to invite as many people to your party as possible. 
    But, there’s a catch: you don’t quite trust the
    other diplomats, all of whom speak multiple languages. 
    So, you’d like to make sure that you can understand
    what everyone is saying at your party. As the Canadian 
    ambassador to Svenborgia, you speak English,
    French, and Svenborgian. You want to make sure 
    that no two of your party guests speak the same language,
    other than the three you speak. For each diplomat, 
    you have a list of of every “foreign” language (i.e.,
    other than English, French, or Svenborgian) that they 
    speak. We refer to this as the International Party
    Guest, or IPG, problem.\\\\
    For example: suppose there 
    are three diplomats. If 
    diplomat 1 speaks language 
    1, diplomat 2 speaks
    languages 2 and 3, and diplomat 
    3 speaks languages 1, 3, and 4, we 
    would represent this instance as
    \{\{1\}, \{2, 3\}, \{1, 3, 4\}\}. 
    The optimal solution to this instance 
    is to invite diplomat 1 and diplomat 2.\\\\
    Consider the four following greedy algorithms 
    designed to maximize the number of guests you 
    can invite. For each, determine whether the 
    algorithm is optimal. If it is, briefly sketch a 
    proof of optimality. If
    it’s not, give a counterexample(next page).
    \newpage
    \begin{enumerate}
    \item[0.] \textbf{Greedy Strategy A:} if no two diplomats 
    speak the same foreign language, invite them all. 
    Otherwise, invite the diplomat who speaks the fewest 
    languages only if they don't share a language with
    the next diplomat in the invite pool, and recurse.\\\\\\\\\\\\
    \item \textbf{Greedy Strategy B:} if no two diplomats 
    speak the same foreign language, invite them all. 
    Otherwise, remove the diplomat who speaks 
    more than half the possible languages, and recurse.\\\\\\\\\\\\
    \item \textbf{Greedy Strategy C:} if no two diplomats 
    speak the same foreign language, invite them all. Otherwise,
    remove the diplomat who speaks the most languages, and recurse.\\\\\\\\\\\\
    \item \textbf{Greedy Strategy D:} if no two diplomats speak the 
    same foreign language, invite them all. Otherwise,
    invite the diplomat who speaks the fewest languages, 
    remove all other diplomats who share a language
    with the one you just invited, and recurse.
    \newpage
    \item Prove that if $n \in \mathbb{Z}$ and $5n+4$ is even, then $n$ is even using proof by contradiction.
    (\textbf{Motivation: }Proof by contradiction is often used in proving greedy algorithms correct.)
    \newpage 
    \item In mathematics, a sequence of positive real numbers 
    $s_1$, $s_2$,$\dots$ is called \textit{superincreasing} if each element
    in the sequence is greater than the sum of all previous elements in the sequence:
    $$s_{n+1} > \sum\limits_{i=1}^{n}s_i$$
    For example: \{$2,3,7,16,65,321,4546$\} is a superincreasing sequence, 
    but \{$1,1,2,5,15,52,203,877$\} us not a superincreasing sequence.\\\\
    Describe an algorithm that takes as input superincreasing sequence $s_1,\dots,s_n$ and a positive 
    integer $k$, please find a sequence of $s_1,\dots,s_n$ with the sum equal
    to $k$. It is possible and desirable to find an algorithm that can accomplish this
    task in $O(n)$ time using dynamic programming. If you think you've come up with an
    algorithm that can accomplish this task attempt to prove that it is correct.
    \newpage
    \item Look back at the cashier's algorithm you've seen in lecture, which I will denote as $MC$ for 
    ``Make Change". Compactly we can write the algorithm as where:
    $$MC(N) = \min_{i}\{MC(N-S_i)+1\}$$
    where $i \in \{1,2,\dots,m\}$ and $S_1<S_2<\dots<S_m$
    \\
    How many unique subproblems are there for this problem?\\\\\\\\\\\\
    \item How much work do you have to do to go from your subproblems back to your original problem?\\\\\\\\\\\
    \item Using the prior two parts find the worst case runtime of $MC(N)$.\\\\\\\\\\\\
    \item Challenge: Is this a polynomial time algorithm? (HINT: Look back at the subproblems)
    \newpage
    \item 
\end{enumerate}
\end{document}