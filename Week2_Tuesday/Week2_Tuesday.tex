\documentclass[12pt]{article}
\usepackage{amsmath,amssymb,amsthm}
\usepackage{graphicx,mathabx}
\usepackage{xcolor}
\usepackage{tikz}
\usepackage{placeins}
\usepackage{lipsum}
\usepackage{mathtools}
\usepackage[shortlabels]{enumitem}
\usepackage{wrapfig}
\DeclarePairedDelimiter{\ceil}{\lceil}{\rceil}
\begin{document}
\title{TCSS 343 - Week 2 - Tuesday}
\author{Jake McKenzie}
\maketitle
\noindent\centerline{\textbf{Asymptotics with Divide and Conquer}}\\\\\\\\\\\\\\\\
\begin{center}
    ``I am incapable of conceiving infinity, and yet I do not accept finity. I want this adventure that is the context of my life to go on without end". \\$\dots$\\ Simone de Beauvoir
\end{center}
\begin{center}
    ``Man has not given himself the taste for the infinite and the love of what is immortal. These sublime instincts do not arise from a caprice of the will; they have their unchanging foundation in his nature; they exist despite his efforts. He can hinder and deform them, but not destroy them."\\
    $\cdots$\\
    Alexis de Tocqueville
\end{center}
\newpage
\begin{enumerate}
\item[0. ] Prove the theorem below using the techniques of binding the term and splitting the
sum to find a tight bound for the sum. Make sure your proof is complete, concise, clear
and precise.\\
\textbf{Theorem 0: }
$$\sum\limits_{i=1}^{n}i^4\in\Theta(n^5)$$
\newpage
\item Prove the theorem below using the techniques of binding the term and splitting the
sum to find a tight bound for the sum. Make sure your proof is complete, concise, clear
and precise.\\
\textbf{Theorem 1: }
$$\sum\limits_{i=1}^{n}(\log_2{i})^2\in\Theta(n(\log_2{n})^2)$$
\newpage
\item Prove the theorem below using the techniques of binding the term and splitting the
sum to find a tight bound for the sum. Make sure your proof is complete, concise, clear
and precise.\\
\textbf{Theorem 2: }
$$\sum\limits_{i=1}^{\sqrt{n}}i^2 \in\Theta(n^{\frac{3}{2}})$$
\newpage
Consider the task of detecting whether a given array has an element that is repeated in more
than half of the positions of the array. For example. the value 2 is the majority element in
the array $\{3, 2, 5, 2, 3, 2, 7, 2, 2\}$, while the array $\{5, 8, 8, 3, 10, 8, 5, 8\}$ has no majority element.
In the present task you will develop a divide and conquer algorithm for solving this problem,
and you will analyze its running time.\\
\item Write a formal statement of this problem. That is, state the input and
output criteria as precisely as possible.
\newpage
\item  Suppose that an array $a[1 \dots n]$ has an element $v_L$ that occurs in strictly more than half
of the first floor($\frac{n}{2}$) positions of $a$, and an element $v_H$ that occurs in strictly more than half
of the remaining ceiling($\frac{n}{2}$) positions. Argue carefully why if there is an element $v$ of $a$ that
occurs in over half of all $n$ positions of $a$, then $v$ must be either $v_L$ or $v_H$. Note that it
would not be enough for $v$ to occur more times in a than either $v_L$ or $v_H$. We explicitly
require that $v$ occur in strictly more than half of all positions of the entire array.
\newpage
\item Using the information you came to in the prior pages, design a divide and conquer algorithm 
that returns the majority element of an array or the value $-1$ if no such element exists. 
Give detailed tidy psuedocode.
\newpage
\item Prove the theorem below using the techniques of binding the term and splitting the
sum to find a tight bound for the sum. Make sure your proof is complete, concise, clear
and precise.\\
\textbf{Theorem 3: }
$$\sum\limits_{i=1}^{n}\log_2{(\frac{n}{i})}\in\Theta(n)$$
\end{enumerate}
\end{document}










