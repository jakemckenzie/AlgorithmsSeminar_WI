\documentclass[12pt]{article}
\usepackage{amsmath,amssymb,amsthm}
\usepackage{graphicx,mathabx}
\usepackage{xcolor}
\usepackage{tikz}
\usepackage{placeins}
\usepackage{lipsum}
\usepackage[shortlabels]{enumitem}
\usepackage{placeins}
\usepackage[makeroom]{cancel}
\usepackage{mathrsfs}
\usepackage{nicefrac}
\newcommand\tab[1][1cm]{\hspace*{#1}}
\def\blankpage{%
      \clearpage%
      \thispagestyle{empty}%
      \addtocounter{page}{-1}%pdf
      \null%
      \clearpage}
\begin{document}
\title{TCSS 343 - Week 0 - Thursday}
\author{Jake McKenzie}
\maketitle
\noindent\centerline{\textbf{Strong Induction and some Mathematical Review}}\\\\\\\\\\\\
\newpage
\begin{enumerate}
\item Today we will delve deeper into strong induction. You may not 
be asked to do this until later in the quarter yourself, but I want 
to review strong induction more heavily as many of the proofs you will 
see will rely on this technique! 
\\\\
\textbf{Strong Induction}\\
\begin{enumerate}
\item[Step 1:] State Claim: We will show $P(n)$ is true $\forall n$, 
using induction on $n$.
\item[Step 2:]  State Basis: We will show that P(1) is true.
\item[Step 3:]  Inductive Hypothesis: State P(k)
\item[Step 4:]  Inductive Step: Suppose $P(1)$ through $P(k)$ is true, for some integer $k$.
We need to show that $P(k+1)$ is true. 
\end{enumerate}

\newpage
\item Simplify the following algebraic expressions.
\begin{enumerate}
\item $9n*(3n^2+5n+8)$
\end{enumerate}    
\end{enumerate}
\end{document}