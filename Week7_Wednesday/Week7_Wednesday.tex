\documentclass[12pt]{article}
\usepackage{amsmath,amssymb,amsthm}
\usepackage{graphicx,mathabx}
\usepackage{xcolor}
\usepackage{tikz}
\usepackage{placeins}
\usepackage{lipsum}
\usepackage[shortlabels]{enumitem}
\usepackage{placeins}
\usepackage[makeroom]{cancel}
\usepackage{mathrsfs}
\newcommand\tab[1][1cm]{\hspace*{#1}}
\def\blankpage{%
      \clearpage%
      \thispagestyle{empty}%
      \addtocounter{page}{-1}%pdf
      \null%
      \clearpage}
\begin{document}
\title{TCSS 343 - Week 7}
\author{Jake McKenzie}
\maketitle
\noindent\centerline{\textbf{Graph Algorithms and some Greedy Algorithms}}\\\\\\\\\\\\\\\\
\begin{center}
    ``Finding a needle in a haystack is actually quite easy. It's finding a very specific piece of hay in a haystack that happens to be hard." \\$\dots$\\ Avi Wigderson
\end{center}
\begin{center}
    ``The methods given in this paper require no foresight or ingenuity,
    and hence deserve to be called algorithms.'' \\$\dots$\\ Edward R. Moore
\end{center}
\begin{center}
    ``Thus you see, most noble Sir, how this type of solution bears little relationship to
    mathematics, and I do not understand why you expect a mathematician to
    produce it, rather than anyone else." \\$\dots$\\ Euler on the creation of graph theory
\end{center}
\newpage
\noindent There are $n$ trading posts numbered $1$ to $n$, 
as you travel downstream. At any trading post $i$, you can rent 
a canoe to be returned at any of the downstream trading posts $j > i$. 
You are given a cost array $R(i,j)$ giving the cost of these rentals for 
$1 \leq i \leq j \leq n$. We will have to assume that $R(i,i) = 0$ and $R(i,j) = \infty$ 
if $i > j$. For example, withn = $4$, the cost array may look as follows: The rows are the 
sources ($i - s$) and the columns are the destinations ($j$’s).\\
\centerline{\includegraphics[scale = .8]{posts.jpg}}\\
The problem is to find a solution that computes the cheapest sequence of rentals taking 
you from post $1$ all the way down to post $n$. \\\\
0. State the input and output conditions for this problem as precisely (tidy) as possible.\\\\\\\\\\\\\\\\\\\\
1. Why do greedy algorithms fail to solve this problem?
\newpage
\noindent 2. Modify Kruskal's algorithm to find the maximum spanning tree.\\
\fbox{
    \parbox{10cm}{
      KRUSKAL \\
      
      $T = \emptyset$ \\
      FOR each vertex $v \in V$ {\sc Make-Set}($v$)
      
      Sort edges of $E$ in increasing order by weight \\
      FOR each edge $e=(u,v) \in E$ in order DO
      \begin{itemize}
      \item[] IF {\sc Find-Set}($u$) $\neq$ {\sc Find-Set}($v$) THEN
	\begin{itemize}
	\item[] $T = T \cup \{ e \}$
	\item[] {\sc Union-Set}($u,v$)
	\end{itemize}
      \end{itemize}
  }}
  \newpage
 \noindent 3. Modify Prim's algorithm to find the maximum spanning tree.\\
  \fbox{
    \parbox{10cm}{
      PRIM(r)\\
      
      For each $v \in V$ DO
      \begin{itemize}
      \item[] {\sc Insert}($PQ,v,\infty$)
      \end{itemize}
	  {\sc Decrease-Key}$(PQ,r,0)$ \\
	  WHILE $PQ$ not empty DO
          \begin{itemize}
          \item[] $u$ = {\sc Deletemin}($PQ$)
	  \item [] (output edge $(u, visit(u))$ as part of MST)
          \item[] For each $(u,v) \in E$ DO
            \begin{itemize}
            \item[] IF $v \in PQ$ and $w(u,v) < $ key($v$) THEN
            \item[] ~~~~visit[$v$] = $u$
            \item[] ~~~~{\sc Decrease-Key}($PQ,v,w(u,v)$)
            \end{itemize}
          \end{itemize}
  }}
\newpage
\noindent 4. A \textit{looped} tree is a weighted, directed graph build from a binary tree by 
adding an edge from every leaf back to the root. Every edge has non-negative weight.\\
\centerline{\includegraphics{looped.jpg}}
How much time would Dijkstra’s algorithm require to compute the
shortest path between two vertices u and v in a looped tree with n
nodes?\\\\\\\\\\\\\\\\
\noindent 5. Describe and analyze a faster algorithm.
\newpage
\noindent 6. Suppose we are given a directed graph G with weighted edges and two
vertices s and t. Describe and analyze an algorithm to find the shortest path from s to
t when exactly one edge in G has negative weight. [Hint: Modify
Dijkstra’s algorithm. Or don’t.]\\\\\\\\\\\\\\\\\\\\
\noindent 7. Describe and analyze an algorithm to find the shortest path from s to t
when exactly k edges in G have negative weight. How does the running
time of your algorithm depend on k?
\newpage
\noindent With this problem I want to present to you an orientation where the 
greedy algorithm fails and when the algorithm succeeds.\\\\
Imagine we have a wizard that knows a few spells. 
Each spell has 3 attributes: Damage, cooldown time, and a cast time. \\\\
\textbf{Cooldown time:} the amount of time (t) it takes 
before being able to cast that spell again. 
A spell goes on ``cooldown" the moment it begins casting.\\\\
\textbf{Cast time:} the amount of time (t) it takes 
to use a spell. While the wizard is casting 
something another spell cannot be cast and 
it cannot be canceled. \\\\
\textit{The question is:} \textbf{How would you maximize damage 
given different sets of spells?} \\\\
It is easy to calculate the highest damage per cast 
time. But what about in situations where it is better 
to wait then to get ``stuck" casting a low damage 
spell when a much higher one is available...for example consider
the two spells and two actions you can do on those spells totally four actions:\\\\\\
Chill Touch: $100$ damage at a rate of $1$ second per cast with a $10$ second cooldown. \\\\
Mage Hand. $10$ damage at a rate of $4$ second per cast with a $0$ second cooldown.\\\\
Optimal action ordering $\Sigma =\{$Chill Touch, Mage Hand, Wait, Repeat$\}$\\\\
6. Given an arbitary amount of time $t$ what is the maximum amount of spells
we can cast $S$ for some arbitary period?
\newpage
\noindent Now imagine that there is one spell, henceforth called the Eldritch Blast, 
which does a very, very large amount of damage much more than the two current spells you have, has $0$ casting time, and has 
some positive cooldown $n$. If all the other spells do much less damage than the 
Eldritch Blast, it will clearly be optimal to cast the Eldritch Blast every n seconds and 
then optimize the cooldown time with the weaker spells.\\\\
\noindent 7. Why does the greedy algorithm work in this case?\\\\\\\\\\\\\\\\\\\\
Now, assume all the other spells also have cooldown $n$. 
If one optimizes a given n-second downtime with these spells, 
then the same spell-sequence will also be possible in the next 
$n$-second downtime, and so we can assume the solution is $n$-periodic.\\\\
\noindent 8. Why does the greedy algorithm fail in this case?\\\\\\\\\\\\\\\\\\\\ 
\end{document}